% !TeX spellcheck = pt_PT
%
% Capítulo 6
%
\chapter{Conclusão} \label{conclusao}
Este capítulo descreve as conclusões que foram adquiridas do trabalho feito até ao momento.

\section{Recapitulação}\label{sec61}
Como referido na secção 2.1, foi possível, através de reuniões com clubes deste desporto, definir as propriedades principais e secundárias da nossa aplicação. Após estas propriedades estarem definidas e estruturadas, foi gerado o modelo de entidades onde assenta a nossa aplicação (demonstrado no Apêndice A).

Todo o trabalho feito até agora foi maioritariamente do lado da aplicação servidor. Como referido no capítulo 3, foram definidas todas as estruturas, tanto em termos de entidades e informação persistente, como em termos do formato em que a aplicação no geral irá ter acesso a esta informação. Distribuindo os focos principais nas camadas de Modelo, Repositório, Negócio e Controlador, podemos de uma forma organizada separar todo o processo que envolve o caminho desde o \emph{browser} até à base de dados. Com o auxílio das ferramentas apresentadas na secção 2.3, conseguimos organizar toda esta partição e mantê-la consistente. 

Do lado da aplicação cliente, como apresentado no capítulo 4, foram apenas geradas as classes que correspondem às classes de Entidades da aplicação servidor.

Todos os testes feitos foram ao lado da aplicação servidor. Podemos observar o comportamento dos \emph{endpoints} e a persistência dos dados na base de dados.
% !TeX spellcheck = pt_PT
%

\chapter{Aplicação Cliente} \label{cliente}

Este capítulo vai apresentar a nossa solução para o lado da aplicação cliente.

\section{Introdução e Estrutura da Aplicação Cliente} \label{sec41}
A aplicação cliente é a segunda partição da nossa aplicação. É a partição onde se encontra a interface de utilizador, painéis de controlo e alguma lógica de negócio adicional.

No estado atual do nosso projeto, a aplicação cliente ainda não apresenta grande desenvolvimento. Para além do sub-projeto gerado em \emph{Angular}, foram geradas as classes correspondentes às Entidades da aplicação servidor em \emph{TypeScript}.

Podemos observar no troço de código seguinte, como exemplo, a classe \emph{Event.ts} inserida no \emph{package} de classes da nossa Aplicação Cliente.

\begin{verbatim}
import {Profile} from './profile';

export class Event {
constructor(
private id?: number,
private name?: string,
private description?: string,
private date?: Date,
private local?: string,
private profiles?: Profile[]
) {
this.id = id ? id : 0;
this.description = description ? description : '';
this.date = date ? date : new Date(0);
this.local = local ? local : '';
this.name = name ? name : '';
this.profiles = profiles ? profiles : [];}}
\end{verbatim}

Um dos aspetos principais a salientar na implementação dos construtores das Entidades na aplicação cliente é a questão das propriedades poderem ser \emph{nullable}. Organizando os construtores para que todas as propriedades sejam \emph{nullable} enquanto se faz a verificação no corpo do construtor para a ausência destas propriedades, permite-se construir objetos atribuindo valores padrão a todas as propriedades que não existem na altura da criação. Este detalhe de implementação ajuda a gerar objetos vazios sem os problemas que ocorrem frequentemente na manipulação de valores \emph{null}.
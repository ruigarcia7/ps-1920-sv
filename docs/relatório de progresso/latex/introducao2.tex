% !TeX spellcheck = pt_PT
%
%
% Capítulo 5
%
\chapter{Introdução} \label{cap:intro}

Este projeto foi desenvolvido no âmbito de Projeto e Seminário, no semestre de Verão de 2020 da Licenciatura em Engenharia Informática e de Computadores. 
Este capítulo está organizado em três secções que descrevem o enquadramento e objectivo do projeto, assim como a organização do documento.


%
% Secção 1.1
%
\section{Enquadramento\label{key}} \label{sec11}

%
%Texto da secção. Na figura~\ref{fig:logotipo} mostra-se o logótipo do ISEL. Em \cite{wiki:bigdata2019} encontra várias referências para o assunto. O artigo \cite{6547630} é o mais popular conforme indicação do IEEE. Logo a seguir aparece \cite{6824752}. A identificação das referências deve ser melhorada.
Este projeto tem como temática principal o desporto Rugby. Sendo uma atividade desportiva pouco reconhecida ou relevante no contexto da nossa cultura, é notória a falta de ferramentas que proporcionem suporte à pratica desta atividade. Este problema foi apresentado aos diversos elementos do grupo devido a ligações interpessoais entre estes e o desporto, experienciando ativamente e lidando com este problema no próprio quotidiano.

Apesar de existir um grupo de ferramentas que proporcionem uma melhor qualidade na organização e prática desta modalidade, este foi considerado pelos estudantes como um grupo escasso e dispendioso, pelo que foi optado como objectivo deste projeto criar uma aplicação que assente na ideia de ajudar ativamente equipas técnicas desta modalidade desportiva em vários temas relevantes à otimização e melhoria do desempenho da equipa ao longo da época desportiva.

% Colocar uma figura
%\begin{figure}[h]
%\begin{center}
%\resizebox{100mm}{!}{\includegraphics{./figures/logoISEL.png}}
%\end{center}
%\caption{Legenda da figura com o logótipo do ISEL.}\label{fig:logotipo}
%\end{figure}

%Continuação do texto depois do parágrafo que refere a figura.

%
% Secção 1.2
%
\section{Objetivos e Funcionalidades}\label{sec12} 

Esta secção aborda os objetivos e funcionalidades principais e secundários. É de notar que entre a proposta de projeto inicial e o corrente relatório ocorreram diversas reuniões com as equipas técnicas do Belas Rugby Clube e Sporting Clube de Portugal, pelo que é notável alguma diversidade de objetivos entre ambos os documentos.

A ideia chave deste projeto é criar uma aplicação que seja capaz de recolher e analisar estatisticamente dados sobre o desempenho dos jogadores de uma equipa de Rugby, afim de monitorizar aspetos críticos que avaliem não só o estado individual de cada jogador, como o estado atual de toda a equipa. Pressupõe-se que toda a informação recolhida consiga facilitar aspetos chaves do funcionamento de uma equipa desportiva, auxiliando desde a face tática do desporto (aspetos como a constituição da equipa, o plano tático, a otimização de temáticas de treino), assim como a face menos técnica de uma equipa (aspetos como a organização de treinos e jogos, a facilidade de acesso a informação pertinente, entre outros).

Pretende-se aglomerar todos estes aspetos numa única aplicação, que ofereça aos utilizadores, sendo eles atletas ou equipa técnica, uma plataforma onde possam observar os dados aglomerados referentes a cada jogador em particular, os dados referentes a jogos concretos ao longo da época, aceder a planos de treinos físicos propostos pela equipa técnica, e observar uma linha temporal sobre todos os eventos futuros contextuais com a equipa. No que toca à equipa técnica, esta também terá a funcionalidade de gerar jogos, manusear jogadores em contexto destes jogos (conceitos como lista de convocados, jogadores titulares, jogadores suplentes), assim como ter acesso a uma interface gráfica onde seja possível adicionar estatísticas aos atletas, oferecendo uma percepção detalhada do desempenho desse atleta no jogo. 

As funcionalidades são explicadas em mais detalhe na secção 2.2.\\
\\

\section{Organização do documento} \label{sec13}
O restante relatório encontra-se organizado da seguinte forma.\\

\begin{tabular}{ll}
	Capítulo 2 & {\bf Formulação do Problema} \\
	& Formulação e Contextualização do Problema, Propriedades Básicas \\
	& e Arquitetura da Solução.\\
	&\rule{75mm}{0.5pt}\\
	Capítulo 3 & {\bf Aplicação Servidor} \\
	& Aspetos relacionados com a aplicação servidor, como abordagens, \\
	& metodologias e detalhes de implementação.\\
	&\rule{75mm}{0.5pt}\\
	Capítulo 4 & {\bf Aplicação Cliente} \\
	& Aspetos relacionados com a aplicação cliente, como abordagens, \\
	&metodologias e detalhes de implementação.\\
	&\rule{75mm}{0.5pt}\\
	Capítulo 5 & {\bf Testes} \\
	& Testes executados sobre as diversas vertentes do projeto.\\
	&\rule{75mm}{0.5pt}\\
	Capítulo 6 & {\bf Conclusões} \\
	& Recapitulação das observações e conclusões importantes.\\
	&\rule{75mm}{0.5pt}\\
\end{tabular}